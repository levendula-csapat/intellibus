\documentclass[a4paper]{article}

%%%%%%%%%%%%%%%%
%%% PREAMBLE %%%
%%%%%%%%%%%%%%%%

%%% PACKAGES %%%
\usepackage{fontspec}                     % set fonts
	\setmainfont{Junicode}
\usepackage[a4paper,margin=3cm]{geometry} % page layout
\usepackage[svgnames]{xcolor}             % rainbowssss *_*
\usepackage{hyperref}                     % enhanced references (links)
	\hypersetup{colorlinks=true,allcolors=NavyBlue}
\usepackage{fancyhdr}                     % headers & footers
\usepackage{titling}                      % macros \thetitle, \theauthor
\usepackage{graphicx}                     % enhanced graphics support
	\graphicspath{{img/}}
\usepackage[toc]{glossaries}              % glossaries (obv)
	\makeglossaries
	\newglossaryentry{ACC}
{%
	name={ACC (Adaptive Cruise Control)},
	description={A device for road vehicles that automatically adjusts the
	vehicle's speed to maintain a safe distance from the vehicles ahead (and
	behind)}
}

\newglossaryentry{wheelSpeedSensor}
{%
	name={wheel speed sensor},
	description={A sensor capable of measuring a vehicle's speed by
	monitoring its wheel(s)}
}

\newglossaryentry{selfCheck}
{%
	name={self-check},
	description={(of a component or system) designed to monitor itself and
	be capable of determining if not functioning adequately}
}

\newglossaryentry{SPOF}
{%
	name={SPOF (Single Point Of Failure)},
	description={An element of a system which can be considered the
	\emph{weakest link}: if it fails, the whole system goes down}
}

\usepackage{enumitem}                     % enhanced enumerations
\usepackage{tabularx}                     % enhanced tables

%%% OTHER %%%
\setlength\headheight{22.3725pt}

%%% META %%%
\title{Assignment \#1 \\ Requirement analysis}
\author{Levendula}
\date{\today}

%%% ADDITIONAL PACKAGE CONFIG %%%
% fancyhdr
\pagestyle{fancy}
\fancyhf{}
\lhead{\theauthor}
\rhead{\thetitle}
\lfoot{\today}
\rfoot{\thepage}

%%%%%%%%%%%%%%%%%%%%%%%%%%%%%%%%%%%%%%%%%%%%%%%%%%%%%%%%%%%%%%%%%%%%%%%%%%%%%%%%

%%%%%%%%%%%%
%%% BODY %%%
%%%%%%%%%%%%

\begin{document}

\begin{titlepage}
	\begin{center}
		\includegraphics[width=8cm]{logo.jpg}

		\vspace{.2cm}

		\textbf{Budapest University of Technology and Economics} \\
		Faculty of Electrical Engineering and Informatics \\
		Department of Measurement and Information Systems \\

		\vspace{2cm}

		{\huge IT System Design (\texttt{VIMIAC01})}

		\vspace{2cm}

		{\huge \bfseries \thetitle}

		\vspace{.5cm}

		{\Large \theauthor}

		\vspace{.5cm}

		{\Large \today}
	\end{center}

	\vfill{}

	{\large Team members:}

	\vspace{.25cm}

	\begin{tabular}{lll}
		Gálik Annamária &
			\texttt{WGMUO2} &
			ancsi666@gmail.com \\
		Szilágyi Borbála &
			\texttt{COVQ1M} &
			szilagyiborbala8@gmail.com \\
		Péter Bertalan Z. &
			\texttt{QO7CU6} &
			bertalan.peter+uni@bertalanp99.eu
	\end{tabular}

	\vspace{2cm}
\end{titlepage}


\tableofcontents
\clearpage

\section{Our task}

% Ancsi: Nem kell a dolgokat {} közé tenni, csak ritkán (akkor, ha alkalmazni
% akarsz rájuk valami olyan makrót, ami ilyen blokkokra teljesül, pl \scriptsize
% (de ilyet csak kivételes esetekben csinálunk)
% A sorok végére sem kell tenni \\-t, hanem egy sort kihagyva érjük el, hogy új
% paragrafus legyen
% -- Berci

Our job was to do the preparation of installing \gls{driverless}
\gls{autonomous} \gls{vehicle}s to offer \gls{transportation} service to the
workers at a private office park.

We had to identify the \gls{stakeholder}s and model the context of the planned
system, and also to define the high-level requirements, necessary functions and
use cases. We also documented non-trivial terms in a glossary as well as the
ambiguous or contradicting requirements so that we can contact the customer with
well-defined questions.

We had to study the appendices, then we had to look through and discuss the
questions that had arisen.


\section{Ambiguous requirements}

\begin{enumerate}
	\item Interpretation of ‘make sure the buses are not overcrowded’

		Ideas:

		\textit{The buses form a network that is capable of
			intelligently deciding which bus to ‘send’ where, in
			accordance to the number of people \gls{request}ing the
			bus at the \gls{terminal}s, what \gls{route}s other
			buses have planned and how many people are on certain
			buses. This approach is to solve the problem mentioned
			above; overcrowding can be avoided provided there is a
			sufficient number of \gls{vehicle}s at our disposal and
			the buses can \gls{autonomous}ly decide their
			\gls{route}s in such a way that no single bus is full
			while others are almost empty. Essentially, we could
			balance the number of passengers on the buses.}

	\item What does a \gls{request} contain of? Who can publish a
	      \gls{request}?

		Ideas:

		\textit{The \gls{request} must contains the indentification of
			the worker who ordered the bus, the number of passengers
			who will travel, the starting point, the destination,
			and the planned time of the trip. To publish a
			\gls{request} needed an authentication to avoid
			un\gls{authorized} access.}

	\item Can the passenger change his/her destination during the
	      \gls{transportation}?

		Ideas:

		\textit{No, but the passengers can get off the \gls{vehicle}
			before their former destination if they indicate their
			intention in time.}
\end{enumerate}


\section{List of stakeholders}

% Ancsi: Erre van egy külön environment: description
%
% Úgy működik, hogy a \item-ek után írsz egy [] közé valamit, ami a magyarázandó
% szó, majd utána írod is a magyarázatot. pl:
%
% \begin{description}
% 	\item[HTTP] HyperText Transfer Protocol
%	\item[MPLS] MultiProtocol Label Switching
% 	\item[SMTP] Simple Mail Transfer Protocol
% \end{description}
%
% (itt ennél kicsit mást használtam, hogy még szebb legyen, de az egy package
% által működik csak (enumitem))
%
% Persze lehet szebbben formázni, ahogy lent tettem, mert hosszúak a sorok.
%
% -- Berci

\begin{description}[align=right,leftmargin=6cm,style=multiline]
	\item[authorities]
		Experts of \gls{autonomous} \gls{vehicle}s with official
		responsibility for a particular area of activity
		% TODO ez szerintem nem ez, hanem a hatóságok -- Berci

	\item[office workers]
		The people working at the client's company and the main users of
		the \gls{autonomous} buses.

	\item[Health\&Safety Department]
		A group at our company responsible for analyzing the
		potential health and safety risks of our products and
		forming rules and regulations to make sure these risks
		are averted.

	\item[project sponsor]
		An organization that contributes financially to the project.

	\item[project manager]
		A person who takes responsibility for the planning, preparation
		and execution of the project. They are responsible for
		accomplishing the project's objectives.

	\item[architects]
		The people who design new architectures. They are responsible
		for achieving a particular plan or aim.
		% TODO az architect ilyen építész, nem általános emberke
		% (gondolom itt csak a buszmegállóra vontakozik)

	\item[\gls{autonomous} \gls{vehicle} control engineers]
		They are capable of developing new technologies and handling
		problems of \gls{autonomous} \gls{vehicle} transport systems
		taking into account environmental and energy management
		requirements.

	\item[mechanical engineers]
		They work with automotive mechanical systems, and design
		power-producing machines.
		% TODO power-producing machines?

	\item[\gls{mobileApplication} developers]
		They create, maintain and implement \gls{mobileApplication}s
		that meet the requirements of the clients.

	\item[\gls{terminal} developers]
		% TODO one sentence a One sentence about their duty. I don't
		% really know right now, sorry.

	\item[system testers]
		They create testing plans for the system and identify what parts
		of a system can be tested using \gls{automated} tools and what
		require manual testing. This team is also responsible for
		running the tests.

	\item[mechanics]
		People who repair and maintain machinery.

	\item[suppliers]
		The organizations that provide needed elements, products and
		services for the project.

	\item[our company’s management]
		The group of people in charge of our company.

	\item[client's management]
		The group of people in charge of our client's company.
\end{description}



\section{System context}

% TODO diagram


\section{Requirements}

\subsection{Functional requirements}
\begin{tabularx}{\textwidth}{p{.75cm} X}
	F1     & The buses shall get around \gls{autonomous}ly inside the
	         \gls{site}. \\

	F1.1   & The buses shall be capable of transporting passengers around
	         the office park. \\

	F2.2   & The buses shall select their \gls{route} \gls{automatically}
	         based on the current demand. \\

        F1.3   & The buses must not have any permanent personnel on board. \\

	F2     & The buses shall have a dedicated \gls{station} in the park for
	         storage and maintenance. \\

	F3     & The passengers should be able to publish their
	         \gls{transportation} \gls{request}s. \\

	F3.1   & Registered users should be able to publish their \gls{request}s
	         in advance using a \gls{mobileApplication}. \\

	F3.2   & Passengers should be able to publish their \gls{transportation}
	         \gls{request}s through the \gls{terminal}s. \\

	F3.2.1 & Dedicated \gls{terminal}s should be installed in front of every
	         major office building. \\
\end{tabularx}

% TODO: kep beszurasa a diagramrol

\subsection{Performance requirements}
\begin{tabularx}{\textwidth}{p{.75cm} X}
        P1 & The waiting time for a bus may not exceed 30 minutes. \\
        P2 & The buses shall operate between 7 AM and 10 PM every workday. \\
\end{tabularx}

% TODO: kep beszurasa a diagramrol

\subsection{Reliability requirements}
\begin{tabularx}{\textwidth}{p{.75cm} X}
	R1 & The buses shall be able to complete at least 100 km on average
	     between two interventions. \\

	R2 & At least ⅔ of the buses should be operational at all times. \\

        R3 & The buses should balance their load evenly if possible. \\
\end{tabularx}

% TODO: kep beszurasa a diagramrol

\subsection{Safety requirements}
\begin{tabularx}{\textwidth}{p{.75cm} X}
        SA01 & The buses must avoid causing any harm to the passengers. \\

	SA02 & The system should have a database which contains actual
	       information about the condition of the buses. \\

        SA03 & The buses must avoid causing accidents. \\

	SA04 & The buses should be equipped with \gls{sensor}s that can
	       \gls{monitor} road conditions. \\

	SA05 & The buses must successfully complete a 30-day test run before
	       acceptance. \\

        SA06 & The buses may not be overcrowded. \\

        SA07 & Health\&Safety regulations must be fulfilled. \\

        SA08 & The buses may have a \gls{blackBox}. \\

	SA09 & The buses should close their doors only when no passenger is
	       standing in the doorway. \\

	SA10 & The buses should only brake gracefully, except when abrupt
	       braking is the only viable option to avoid an accident. \\

	SA11 & The Health\&Safety Department should have access to the database.
	       \\

	SA13 & The \gls{autonomous} \gls{vehicle}s must be able to react to the
	       behaviour of cyclists and pedestrians. \\
\end{tabularx}

% TODO: kep beszurasa a diagramrol

\subsection{Security requirements}
\begin{tabularx}{\textwidth}{p{.75cm} X}
	SE1 & The system must provide control access only \gls{authorized}
	      personnel. \\

	SE2 & The \gls{mobileApplication} must protect the personal information
	      of the users. \\
\end{tabularx}

% TODO: kep beszurasa a diagramrol

\subsection{Supportability requirements}
\begin{tabularx}{\textwidth}{p{.75cm} X}
	SU1 & The buses should be able to notify the maintenance team and return
	      to the \gls{station} \gls{autonomous}ly in case of a
	      \gls{non-criticalFailure}. \\

	SU2 & Buses may have a group of 2–3 people \gls{station}ed on \gls{site}
	      who can intervene (either remotely or manually) in case of a
	      \gls{technicalProblem}. \\
\end{tabularx}

% TODO: kep beszurasa a diagramrol

\subsection{Usability requirements}
\begin{tabularx}{\textwidth}{p{.75cm} X}
        U1 & The buses must be able to operate under any weather conditions. \\

        U2 & The bus \gls{station}s may be covered. \\

	U3 & The \gls{mobileApplication}’s login screen should be simple and
	     efficient. \\
\end{tabularx}

% TODO: kep beszurasa a diagramrol


\section{Use cases}

\begin{description}[style=nextline]
	\item[Make \gls{request} via \gls{mobileApplication}]
		Use the \gls{mobileApplication} designed especially for this
		system to make a \gls{request} for a lift. Only registered users
		may make \gls{request}s this way. The \gls{request} includes the
		user's identification, the date and time of the travel, their
		starting and destination point as well as the number of people
		who wish to travel in case of a group \gls{request}.

	\item[Make \gls{request} via \gls{terminal}]
		Use the installed \gls{terminal}s in front of every office
		building to make a tranportation \gls{request}. When making a
		\gls{request} this way, only the travel destionation can (and
		must) be specified.

	\item[Monitor system status]
		Periodically check if the entire system is operating as expected
		— ie an appropriate number of buses is in operation (with regard
		to the number of users at the moment), the buses calculate
		\gls{route}s correctly and do not develop any faults. It must
		also be checked that the platforms for making \gls{request}s can
		be published are fully operational.

	\item[Monitor bus condition]
		Keep watch on the various parameters of a bus to make sure it is
		functioning adequately.

	\item[Monitor bus position]
		Keep track of the positions of the buses to make sure they are
		located where they are supposed to be at any given time.

	\item[Avoid causing accidents]
		Ensure the buses cause harm to neither their environment nor
		their passengers. This is largely accomplished by collecting
		data from a variety of \gls{sensor}s and reacting in every
		situation intelligently and handling unexpected occurrences.

	\item[Drive bus]
		Control the movement of a bus in order to fulfill the main
		purpose of the system: to transport passengers.

	\item[Find alternative \gls{route}]
		When a bus cannot run on a given \gls{route} (which can occur
		due to a variety of reasons) it must find an alternative
		\gls{route} to reach its destination provided such a path
		exists. The buses are aware of the road layout of the office
		park.

	\item[Select \gls{route}]
		The act of choosing a path between two locations in the office
		park, preferably one that is the shortest possible.

	\item[Transport passengers]
		Get passengers to their desired destinations.

	\item[Manage \gls{request}s]
		Take \gls{request}s from users and use them to \gls{route} the
		buses in such a way that the \gls{request}s are carried out
		efficiently and quickly.

	\item[Monitor environment]
		The buses are aware of their surrounding area using several
		\gls{sensor}s installed for this purpose.

	\item[Notify mechanics]
		In case of a fault the bus ‘calls’ for repairs. Repairs can be
		made either remotely or manually. When possible, the buses
		return to the nearest \gls{terminal} where such repairs can be
		made.

	\item[Repair and maintain]
		Fix any potential problems with a bus and keep it in shape for
		continued usage.

	\item[Travel]
		Use the system to get from point A to point B.
\end{description}


\section{Work journal}

% TODO copy from Sheets when done

\clearpage
\printglossaries

\end{document}
