\documentclass[a4paper]{article}

%%%%%%%%%%%%%%%%
%%% PREAMBLE %%%
%%%%%%%%%%%%%%%%

%%% PACKAGES %%%
\usepackage{fontspec}                     % set fonts
	\setmainfont{Junicode}
\usepackage[a4paper,margin=3cm]{geometry} % page layout
\usepackage[svgnames]{xcolor}             % rainbowssss *_*
\usepackage{hyperref}                     % enhanced references (links)
	\hypersetup{%
		colorlinks=true,
		allcolors=NavyBlue}
\usepackage{fancyhdr}                     % headers & footers
\usepackage{titling}                      % macros \thetitle, \theauthor
\usepackage{graphicx}                     % enhanced graphics support
	\graphicspath{{img/}}
\usepackage[toc]{glossaries}              % glossaries (obv)
	\setglossarystyle{altlisthypergroup}
	\makeglossaries
	\newglossaryentry{ACC}
{%
	name={ACC (Adaptive Cruise Control)},
	description={A device for road vehicles that automatically adjusts the
	vehicle's speed to maintain a safe distance from the vehicles ahead (and
	behind)}
}

\newglossaryentry{wheelSpeedSensor}
{%
	name={wheel speed sensor},
	description={A sensor capable of measuring a vehicle's speed by
	monitoring its wheel(s)}
}

\newglossaryentry{selfCheck}
{%
	name={self-check},
	description={(of a component or system) designed to monitor itself and
	be capable of determining if not functioning adequately}
}

\newglossaryentry{SPOF}
{%
	name={SPOF (Single Point Of Failure)},
	description={An element of a system which can be considered the
	\emph{weakest link}: if it fails, the whole system goes down}
}

\usepackage{enumitem}                     % enhanced enumerations
\usepackage{tabularx}                     % enhanced tables
\usepackage{float}                        % 'H' figure placement
\usepackage{rotating}                     % sidewaysfigure
\usepackage[noabbrev]{cleveref}           % clever references
\usepackage{booktabs}                     % pretty tables
\usepackage{tikz}                         % drawings
	\usetikzlibrary{shapes}

%%% OTHER %%%
\setlength\headheight{22.3725pt}

%%% META %%%
\title{Assignment \#3 \\ Behavioural Modelling}
\author{Levendula}
\date{\today}

%%% ADDITIONAL PACKAGE CONFIG %%%
% fancyhdr
\pagestyle{fancy}
\fancyhf{}
\lhead{\theauthor}
\rhead{\thetitle}
\lfoot{\today}
\rfoot{\thepage}

%%%%%%%%%%%%%%%%%%%%%%%%%%%%%%%%%%%%%%%%%%%%%%%%%%%%%%%%%%%%%%%%%%%%%%%%%%%%%%%%

%%%%%%%%%%%%
%%% BODY %%%
%%%%%%%%%%%%

\begin{document}

\begin{titlepage}
	\begin{center}
		\includegraphics[width=8cm]{logo.jpg}

		\vspace{.2cm}

		\textbf{Budapest University of Technology and Economics} \\
		Faculty of Electrical Engineering and Informatics \\
		Department of Measurement and Information Systems \\

		\vspace{2cm}

		{\huge IT System Design (\texttt{VIMIAC01})}

		\vspace{2cm}

		{\huge \bfseries \thetitle}

		\vspace{.5cm}

		{\Large \theauthor}

		\vspace{.5cm}

		{\Large \today}
	\end{center}

	\vfill{}

	{\large Team members:}

	\vspace{.25cm}

	\begin{tabular}{lll}
		Gálik Annamária &
			\texttt{WGMUO2} &
			ancsi666@gmail.com \\
		Szilágyi Borbála &
			\texttt{COVQ1M} &
			szilagyiborbala8@gmail.com \\
		Péter Bertalan Z. &
			\texttt{QO7CU6} &
			bertalan.peter+uni@bertalanp99.eu
	\end{tabular}

	\vspace{2cm}
\end{titlepage}


\tableofcontents
\listoffigures
\clearpage



% ------------------------------------------------------------------------------
\section{Our task}

We were assigned to model the behaviour of some components related to boarding
using state machine and activity diagrams.

First, we modelled the operation of the bus's doors by means of a state machine
diagram. The behaviour of the doors was already defined in some specific cases
on sequence diagrams, which we made sure to conform to.

Secondly, we created an activity diagram describing the \gls{RFIDScanner} based
on the textual user manual.

As a final task, we made another state machine diagram which models the
behaviour of the \gls{boardingController} based on the other components'
behaviour—diagrams for some components have been given to us, such as the
\gls{turnstile}'s state machine diagram. We made sure to adhere to the critical
requirements.


% ------------------------------------------------------------------------------
\section{Solutions}


% - - - - - - - - - - - - - - - - - - - - - - - - - - - - - - - - - - - - - - -
\subsection{The state machine representation of the door's behaviour}

The door intutively has two states: \texttt{Opened} and \texttt{Closed}. We have
added an additional \texttt{Closing} state to model the situation when the
\gls{boardingController} has instructed the door to close, but it is blocked by
something (eg a passenger). In this case, the door keeps trying to close until
successful.

Successfully transitioning into the \texttt{Opened} and \texttt{Closed} states
results in respective \texttt{Opened} and \texttt{Closed} signals being sent.

For the sake of completeness, we also added transitions that loop back to the
same state (for example upon receiving an \texttt{Open} signal in the
\texttt{Opened} state).

See \cref{fig:stm-door} for the state machine diagram.

\begin{figure}[p]
	\includegraphics[width=\textwidth]{stm-DoorBehavior.jpg}
	\caption{State machine diagram modelling the door's behaviour}%
	\label{fig:stm-door}
\end{figure}


% - - - - - - - - - - - - - - - - - - - - - - - - - - - - - - - - - - - - - - -
\subsection{Activity diagram for the RFID scanner}

% TODO


% - - - - - - - - - - - - - - - - - - - - - - - - - - - - - - - - - - - - - - -
\subsection{Boarding contoller behaviour}

% TODO



% ------------------------------------------------------------------------------
\section{Work journal}

\begin{tabularx}{\textwidth}{l l l X}
	\toprule
	Team member & Date & Time\footnote{in hours} & Activity \\ \midrule

	% TODO
	\bottomrule
\end{tabularx}

\clearpage
\glsaddall
\printglossaries

\end{document}
